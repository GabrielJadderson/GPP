\section{Testing}
All testing described in this section is done in the topology described in the project description, which is also included in the introduction as figure \ref{fig:GivenTopology}.\\
\\
The hosts A and B send each other a number of messages in the format "HEH:\#\#\#\textbackslash0". The routers these messages are sent to can be determined by choosing the routing table with argument 2.\\
The numbering starts at 1 and goes up incrementally, so it can be verified that they all arrive in the correct order and without duplicates.\\
\\
The actual tests have been made with all four sets of routing tables and with 100 messages both ways. This means that full-duplex communication has been tried going through both routers, and half-duplex through both routers simultaneously for both "circling" directions. (Referring to the shape of the topology)\\
\\
The testing was done with the command './network -pmynetwork -e250 -n4 -a 5 \#', where \# is the set of routing tables to run with. The testing was done for all valid routing tables with 100 messages.\\
\\
\subsection{Logs for Routing Table Set 0 Hosts}
Here are the logs for routing table set 0 for the hosts. The complete log is in the appendix.\\
\\
Log for station 1: (Host A)
\begin{lstlisting}[breaklines=true]
Station 1: arg2 = 5
Starting network simulation
Station 1: arg3 = 0
[Station 1 ready]
GO received! - error freq 0.250
Starting process (selective_repeat)
Starting process (networkLayerHost)
Starting process (fake_transportLayer)
1 SUCCES:  fake_transportLayer SUCCES:  TL: Received from host with address 212: 'HEH:  1'
1 SUCCES:  fake_transportLayer SUCCES:  TL: Received from host with address 212: 'HEH:  2'
1 SUCCES:  fake_transportLayer SUCCES:  TL: Received from host with address 212: 'HEH:  3'
1 SUCCES:  fake_transportLayer SUCCES:  TL: Received from host with address 212: 'HEH:  4'
1 SUCCES:  fake_transportLayer SUCCES:  TL: Received from host with address 212: 'HEH:  5'
1 SUCCES:  fake_transportLayer SUCCES:  TL: Received from host with address 212: 'HEH:  6'
1 SUCCES:  fake_transportLayer SUCCES:  TL: Received from host with address 212: 'HEH:  7'
1 SUCCES:  fake_transportLayer SUCCES:  TL: Received from host with address 212: 'HEH:  8'
1 SUCCES:  fake_transportLayer SUCCES:  TL: Received from host with address 212: 'HEH:  9'
1 SUCCES:  fake_transportLayer SUCCES:  TL: Received from host with address 212: 'HEH: 10'
<<stop-signal received.>>
<<Sending stop-signal>>
>> Press enter to terminate!
\end{lstlisting}

Log for station 2: (Host B)
\begin{lstlisting}[breaklines=true]
Station 2: arg2 = 5
Starting network simulation
Station 2: arg3 = 0
[Station 2 ready]
GO received! - error freq 0.250
Starting process (networkLayerHost)
Starting process (fake_transportLayer)
Starting process (selective_repeat)
2 SUCCES:  fake_transportLayer SUCCES:  TL: Received from host with address 111: 'HEH:  1'
2 SUCCES:  fake_transportLayer SUCCES:  TL: Received from host with address 111: 'HEH:  2'
2 SUCCES:  fake_transportLayer SUCCES:  TL: Received from host with address 111: 'HEH:  3'
2 SUCCES:  fake_transportLayer SUCCES:  TL: Received from host with address 111: 'HEH:  4'
2 SUCCES:  fake_transportLayer SUCCES:  TL: Received from host with address 111: 'HEH:  5'
2 SUCCES:  fake_transportLayer SUCCES:  TL: Received from host with address 111: 'HEH:  6'
2 SUCCES:  fake_transportLayer SUCCES:  TL: Received from host with address 111: 'HEH:  7'
2 SUCCES:  fake_transportLayer SUCCES:  TL: Received from host with address 111: 'HEH:  8'
2 SUCCES:  fake_transportLayer SUCCES:  TL: Received from host with address 111: 'HEH:  9'
2 SUCCES:  fake_transportLayer SUCCES:  TL: Received from host with address 111: 'HEH: 10'
<<Sending stop-signal>>
<<stop-signal received.>>
>> Press enter to terminate!
\end{lstlisting}
