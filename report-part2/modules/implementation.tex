\section{Implementation}

\subsection{Refactoring}

We've cleaned and refactored the code from the previous project and we've removed most of the locks as they can only be accessed by their context thread.
Furtheremore, events have been placed in their consecutive class.
The following elements have been moved to $Events.h$
\break
\\
\texttt{network\_layer\_allowed\_to\_send}
\break
\texttt{network\_layer\_ready}
\break
\texttt{data\_for\_network\_layer}
\\

We've added the following two signals/events to the $Events.h$

\begin{lstlisting}
#define NL_SendingQueueOffer    0x00000020 //TL->NL: Offer NL a queue of elements for NL to forward. The offered concurrentFifoQueue is assumed locked.
#define TL_ReceivingQueueOffer  0x00000040 //NL->TL: Offer TL a queue of elements for it to receive. The offered concurrentFifoQueue is assumed locked.
\end{lstlisting}


\subsection{ConcurrentFifoQueue}

We've implemented a concurrent FIFO queue, the idea is to have a shared queue between the layers, such as between TL and NL.
The concurrency is provided via a simple true/false value. The purpose of the queues is to be offered to another layer, and as such should be "locked" by setting the used value to a non-zero value and then signaling the layer with a pointer to this whole structure. The receiver of the offer must then "unlock" the queue by setting the used value to 0. With this set of use rules, it is a concurrently safe implementation.
\begin{lstlisting}
typedef struct {
  int used;
  FifoQueue queue;
} ConcurrentFifoQueue;
\end{lstlisting}



\subsection{Datagram and Payload}

\begin{lstlisting}
// Datagram type. Explicitly not typedeffed. More can be added.
enum {DATAGRAM, ROUTERINFO};
/* Documentation:
DATAGRAM: A packet to be delivered to the transport layer.
ROUTERINFO: Router information request, if own addres == dest, send a DATAGRAM to src with information based on the contents of the payload. Unimplemented.
*/
\end{lstlisting}


\subsubsection{Payload}
The following is the implementation of the payload, we've chosen a compile-time value to define the \texttt{MAX\_PAYLOAD} size (8 bytes)
\begin{lstlisting}
#define MAX_PAYLOAD 8
typedef struct { char data[MAX_PAYLOAD]; } payload;
\end{lstlisting}

\subsubsection{Datagram}

Our networkAddress is defined as a 32-bit address similarly to IPv4
\begin{lstlisting}
typedef signed long networkAddress;
\end{lstlisting}


Our datagram as specified in the \ref{sec:DATAGRAM} Design section.

\begin{lstlisting}
typedef struct  {
  //First 32bit word
  unsigned char type;         //8-bit
  unsigned char reserved;     //Currently unused. Exists for expandability and to pad to 32-bit word. Considering a next-header value with additional options as in IPv6.
  unsigned short payloadsize; //16-bit

  //Two next 32-bit words
  networkAddress src;         //source address
  networkAddress dest;        //destination address

  payload payload;            //Payload
} datagram;
\end{lstlisting}

\subsection{Network Layer}

The following is the implmenetation of \texttt{NL\_RoutingTable}, the routing table is used to look up hosts and routers in the network.
Here a compile-time value is used, but in the real world a completely different implementation would be used. This one is used for the sake of simplicity as the network is of minimal size.
%Here we've defined a static variable, in the real world the \texttt{NL\_ROUTING\_TABLE\_SIZE} should be incremented and decremented dynamically. %This would be done in a completely different fashion in the real world.
\begin{lstlisting}
//Routing table. There can be more elements in this table than there are neighbours and multiple entries can point to the same neighbour.
#define NL_ROUTING_TABLE_SIZE 4
typedef struct {
  networkAddress addresses[NL_ROUTING_TABLE_SIZE];
  neighbourid    neighbourids[NL_ROUTING_TABLE_SIZE];
} NL_RoutingTable;
\end{lstlisting}


%Similarly \texttt{NL\_OfferElement} is used in the communication between LL and NL.
In the communication between NL and LL, \texttt{NL\_OfferElement} data elements are used. This allows NL to specify outbound neighbour for the datagram.
\begin{lstlisting}
typedef struct {
  neighbourid otherHostNeighbourid;
  datagram dat;
} NL_OfferElement;
\end{lstlisting}

%\linebreak

\subsection{Network Layer: Hosts}

%The following is the implementation of \texttt{networkLayerHost()} with all unimportant parts ommited.
The following are the most significant parts of  \texttt{networkLayerHost()}, which is the implementation of the network layer for the hosts.

\begin{lstlisting}
  TL_OfferElement *o;
  NL_OfferElement *O; //Using a pointer to create a new one every time.
  datagram d;
  ...
  switch(event.type) {
    ...
    case data_for_network_layer:
      d = *((datagram*) (event.msg));              //retrieve our datagram
      ...
      switch(d.type) {
        case DATAGRAM:
          o = malloc(sizeof(TL_OfferElement));     //assign little o as TL_OfferElement
          o->otherHostAddress = d.src;             //assign address
          o->seg = d.payload;                      //assign payload

      }
      ...
    case NL_SendingQueueOffer:    //NL_OfferSendingQueue:
      ...
      while(EmptyFQ(offer->queue) == 0) {         //while not empty
        e = DequeueFQ(offer->queue);              //pop an element
        o = ((TL_OfferElement*) ValueOfFQE(e));   //assign it to TL_OfferElement

        d.src = thisNetworkAddress;               //resolve source address
        d.dest = o->otherHostAddress;             //resolve destination address

        d.payload = o->seg;  //get the segment from TL_OfferElement as our payload.



        O = malloc(sizeof(NL_OfferElement));      //assign big O as NL_OfferElement
        O->otherHostNeighbourid = NL_TableLookup(d.dest); //get address from lookup table.
        O->dat = d;                        //store out datagram in NL_OfferElement
        EnqueueFQ(NewFQE((void *)O), sendingBuffQueue); //send NL_OfferElement
      }

      ...

      ...
  }
\end{lstlisting}



\subsubsection{Network Layer: Routers}

The following is the implementation of \texttt{networkLayerRouter()}% which \ref{design} %???
\begin{lstlisting}
void networkLayerRouter() {
  event_t event;
  boolean allowedToSendToLL = false;
  FifoQueue receivedQueue = InitializeFQ();
  ConcurrentFifoQueue routersendingQueue = CFQ_Init();
  NL_OfferElement *O;
  datagram d;

  while(true) {
    Wait(&event, events_we_handle);

    switch(event.type) {
      case network_layer_allowed_to_send:
        allowedToSendToLL = true;
        break;
      case data_for_network_layer:
        logLine(succes, "NL: Received datagram from LL.\n");

        d = *((datagram*) (event.msg));

        if(d.dest != thisNetworkAddress) {
          logLine(succes, "NL: Packet was not addressed for this Router. Forwarding.\n");

          O = malloc(sizeof(NL_OfferElement));
          O->otherHostNeighbourid = NL_TableLookup(d.dest); //Lookup the destination
          O->dat = d;

          EnqueueFQ(NewFQE((void *)O), receivedQueue);
          break; //Done.
        }
        break;
    }

    if(routersendingQueue.used == false && EmptyFQ(receivedQueue) == 0) {
      logLine(trace, "NL: Transfering between queues\n");
      //Transfer from one queue to the other.
      while(EmptyFQ(receivedQueue) == 0) {
        EnqueueFQ(DequeueFQ(receivedQueue), routersendingQueue.queue);
      }
    }

    //If we are allowed to send, then do so.
    // If the signal was sent in this loop iteration: same as if this was in the case directly
    // otherwise: we received something to send after getting clearance and would have been stuck if this was directly in the case.
    if(allowedToSendToLL && EmptyFQ(routersendingQueue.queue) == 0 && routersendingQueue.used == false) {

      logLine(trace, "NL: Offering queue to LL\n");

      LL_OfferSendingQueue(&routersendingQueue);
      allowedToSendToLL = false;
    }
  }
}
\end{lstlisting}



\subsection{Transport Layer}

%\texttt{TL\_OfferElement} is used by the TL to figure out where to process the segment %Wot
\texttt{TL\_OfferElement} is used in the communication between TL and NL, where TL sets the otherHostAddress value to be the address of the receiving host and NL sets this value to be the address of the sending host.
\begin{lstlisting}
typedef struct {
  networkAddress otherHostAddress;
  payload seg;
} TL_OfferElement;
\end{lstlisting}
%\break

The implmenetation of the fake transportlayer for testing the network layer is as follows:

\begin{lstlisting}
void fake_transportLayer() {
  ConcurrentFifoQueue *offer;
  ConcurrentFifoQueue *q = malloc(sizeof(ConcurrentFifoQueue));
  *q = CFQ_Init();

  long int events_we_handle = TL_ReceivingQueueOffer;
  event_t event;

  FifoQueueEntry e;
  TL_OfferElement o;

  TL_OfferElement *elem;

  #define TL_NUM_HEH 10
  int i = 1;
  while(i <= TL_NUM_HEH) {
    elem = malloc(sizeof(TL_OfferElement));

    elem->otherHostAddress = 42;
    if(ThisStation == 1) {
      elem->otherHostAddress = 212;
    } else if(ThisStation == 2) {
      elem->otherHostAddress = 111;
    }

    char a = i/100;
    if(i < 100) {a = ' ';} else {a += 48;}
    char b = (i%100)/10;
    if(i < 10) {b = ' ';} else {b += 48;}
    char c = (i%10)+48;

    elem->seg.data[0] = 'H';
    elem->seg.data[1] = 'E';
    elem->seg.data[2] = 'H';
    elem->seg.data[3] = ':';
    elem->seg.data[4] =  a ;
    elem->seg.data[5] =  b ;
    elem->seg.data[6] =  c ;
    elem->seg.data[7] = '\0';

    EnqueueFQ( NewFQE( (void *) elem ), q->queue );
    i++;
  }


  logLine(trace, "TL: Offering queue.\n");
  NL_OfferSendingQueue(q); //We can do this whenever really as it isn't based on a 1:1 signals-handing-out-information thing.
  int numReceivedPackets = 0;

  while(true) {
    logLine(trace, "TL: Waiting for signals.\n");
    Wait(&event, events_we_handle);

    switch(event.type) {
      case TL_ReceivingQueueOffer:
        logLine(debug, "TL: Received signal TL_ReceivingQueueOffer.\n");
        offer = (ConcurrentFifoQueue*) event.msg;
        logLine(trace, "TL: Received offer queue.\n");
        //Assumes that the queue has already been locked by the offerer. The offerer intends only this process to access the queue and renounces its own access.

        //Handle incoming messages
        while(EmptyFQ(offer->queue) == 0) {
          logLine(trace, "TL: Handling element.\n");
          e = DequeueFQ(offer->queue);
          o = *((TL_OfferElement*) (e->val));

          logLine(succes, "TL: Received from host with address %d: '%s'\n", o.otherHostAddress, o.seg.data);
        }

        logLine(trace, "TL: Releasing locks.\n");
        numReceivedPackets++;
        break;
    }

    if(numReceivedPackets >= TL_NUM_HEH) {
      sleep(2);
      Stop();
    }
  }
}

\end{lstlisting}




\hfill \break
