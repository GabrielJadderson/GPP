\section{Implementation}
TODO: INTRO TO IMPL
\\
\subsection{Overview}

\begin{lstlisting}
typedef struct {
    int stationID;
		int ack_timer_id;
    int timer_ids[NR_BUFS];
    boolean no_nak;
} neighbour;
\end{lstlisting}

Each neighbour will have it's own associated stationID,
this way we can differentiate between each neighbour/station.
Furtheremore each neighbour has it's own \texttt{ack\_timer\_id} that way we ensure that there's only one instance per neighbour
Each neighbour will start it's own ack timer when they recieve a pakcet,
this way we can ensure the piggy-backing remains functional.
\\

The following struct is only used for the selective-repeat function,
encapsulating our variables in a struct proved to be much more manageable.

\begin{lstlisting}
typedef struct { //The data only visible to selective repeat.
  seq_nr ack_expected;              // lower edge of sender's window
  seq_nr next_frame_to_send;        // upper edge of sender's window + 1
  seq_nr frame_expected;            // lower edge of receiver's window
  seq_nr too_far;                   // upper edge of receiver's window + 1
  //int i;                            // index into buffer pool //Was never used?
  //frame r;                          // scratch variable //Not needed to have it in here, it is used at the outermost level of SR
  packet out_buf[NR_BUFS];          // buffers for the outbound stream
  packet in_buf[NR_BUFS];           // buffers for the inbound stream
  boolean arrived[NR_BUFS];         // inbound bit map
  seq_nr nbuffered;                 // how many output buffers currently used
} neighbour_SR_Data;
\end{lstlisting}


We have further implemented \texttt{packetTimerMessage} a struct that is sent to the library whenever we start a timer.
This way we know which sequence the timer was started for and the respective neighbour id.
\begin{lstlisting}
typedef struct {
  seq_nr k;
  neighbourid neighbour;
} packetTimerMessage;
\end{lstlisting}

\hfill \break
\subsection{Actual Code}




We've changed the following functions to take the neighbours id's as parameters.
Inside the function we've used the parameter to retrieve the neighbour specified from a global neighbours array.
This allows us to stop and start the timers for each respective neighbour.

\begin{lstlisting}
void start_timer(neighbourid neighbour, seq_nr k)
void stop_timer(neighbourid neighbour, seq_nr k)
void start_ack_timer(neighbourid neighbour)
void stop_ack_timer(neighbourid neighbour)
\end{lstlisting}


Our implementation of \texttt{start\_timer} is as follows. We first allocate memory for our {packetTimerMessage}
The sequence number is updated accordingly to our parameter, on the \texttt{start\_ack\_timer} function however, we set the sequence number to -1 (not used) since it redundant in that case.
Afterwards the neighbour id is updated accordingly and a timer is initiated from the SetTimer function, this function starts a timer and returns the id of that timer,
which we then store in our specified neighbour.
we use modulo to wrap it around the index range since the sequence numbers increase linearly.
\begin{lstlisting}
void start_timer(neighbourid neighbour, seq_nr k) {
        packetTimerMessage *msg = malloc(sizeof(packetTimerMessage));
        msg->k = k;
        msg->neighbour = neighbour;

	neighbours[neighbour].timer_ids[k % NR_BUFS] = SetTimer( frame_timer_timeout_millis, (void *)msg );
	logLine(trace, "start_timer for seq_nr=%d timer_ids=[%d, %d, %d, %d] %s\n", k, neighbours[neighbour].timer_ids[0], neighbours[neighbour].timer_ids[1], neighbours[neighbour].timer_ids[2], neighbours[neighbour].timer_ids[3], msg);
}
\end{lstlisting}


To stop the timers we simply invoke the subnet function: \texttt{StopTimer} with the id of the timer
and in return we get the associated **msg. (note: this is why we store the id of thet timer in the first place)


The following is a description of what the three events in our code do:
\\

%\begin{itemize}
%  \item[$\square$]{ \texttt{ network\_layer\_allowed\_to\_send:}\\
%  Signals the network layer that it can send a piece of data. Here the network layer should make sure the \texttt{from\_network\_layer\_queue} contains at least one element, after which it can signal \texttt{network\_layer\_ready}, which means that the network layer has prepared at least one element in the queue.}
%
%  \item[$\square$]{ \texttt{ network\_layer\_ready:}\\
%  This signal signals to the link layer that the network layer has prepared an element to be sent in the
%  \texttt{from\_network\_layer\_queue} queue and that it should be sent now.}
%
%  \item[$\square$]{ \texttt{ data\_for\_network\_layer:}\\
%  This signal signals to the network layer that the \\
%  \texttt{for\_network\_layer\_queue} contains a data element for it to take care of. (a data element has been received)}
%\end{itemize}

\begin{description}[leftmargin=1em, style=nextline]
\item [\texttt{network\_layer\_allowed\_to\_send}] Signals the network layer that it can send a piece of data.
Here the network layer should make sure the \texttt{from\_network\_layer\_queue}
contains at least one element, after which it can signal \texttt{network\_layer\_ready},
which means that the network layer has prepared at least one element in the queue.

\item [\texttt{network\_layer\_ready}] This signal signals to the link layer that the network layer has prepared an element to be sent in the
  \texttt{from\_network\_layer\_queue} queue and that it should be sent now.

\item [\texttt{data\_for\_network\_layer}]
This signal signals to the network layer that the
\texttt{for\_network\_layer\_queue} contains a data element for it to take care of. (a data element has been received)
\end{description}

\hfill \break
