\section{Conclusion}
Our challenges were to fully understand what was required of us to do, therefore, we had to read the project description several times.
our initial thoughts were to create a two-dimensional array for each neighbour with it's timers and buffers, we decided to go with structs as
that would provide us with a greater foundation later on, it also required us to read the rest of the code and not just replace variables with arrays.
we started implmenting the structs and replacing the function parameters accordingly.

We then imidiately started testing, the testing was very straightforward,
we figured the run command -exxx simulates a packet loss scenario and -nx denotes the amount of neighbours/stations to execute.
we've tested with different packet loss scenarios and with machines more than 2,
and as seen from our testing above we can conclude that the program satisfies the requirements for the program and works as intended.
