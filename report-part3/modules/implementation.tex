\section{Implementation}
\subsection{Part 1}
TODO: implement
\subsection{Part 2}
We've added the following two signals to the $Events.h$
$NL\_SendingQueueOffer$
and
\break
$TL\_ReceivingQueueOffer$
We've also implemented a concurrent FIFO queue, the idea is to have a shared queue between TL and the NL.

\subsubsection{Datagram and Payload}
Datagrams is used for handling messages from/to the transport layer.
The implementation of the datagram and payload is as follows.
\begin{lstlisting}
typedef struct  {
  //First 32bit word
  unsigned char type;         //8-bit
  unsigned short payloadsize; //16-bit
  //Two next 32-bit words
  networkAddress src;         //source address
  networkAddress dest;        //destination address
  payload payload;            //Payload
} datagram;
#define MAX_PAYLOAD 8
typedef struct { char data[MAX_PAYLOAD]; } payload;
\end{lstlisting}
The networkAddress is defined as a 32-bit address similarly to IPv4.

\subsection{Network Layer}

The following is the implementation of \texttt{NL\_RoutingTable}, the routing table is used to look up hosts and routers in the network.
Here we've defined a static variable, in the real world the \texttt{NL\_ROUTING\_TABLE\_SIZE} should be incremented and decremented dynamically.
There can be more elements in Routing table than there are neighbours and multiple entries can point to the same neighbour.
\begin{lstlisting}
#define NL_ROUTING_TABLE_SIZE 4
typedef struct {
  networkAddress addresses[NL_ROUTING_TABLE_SIZE];
  neighbourid    neighbourids[NL_ROUTING_TABLE_SIZE];
} NL_RoutingTable;
\end{lstlisting}


Similarly \texttt{NL\_OfferElement} is used in the communication between LL and NL.
\begin{lstlisting}
typedef struct {
  neighbourid otherHostNeighbourid;
  datagram dat;
} NL_OfferElement;
\end{lstlisting}

The following is the implementation of the signal handling in \texttt{networkLayerHost()} with all unimportant parts ommited.
\begin{lstlisting}
  switch(event.type) {
    case data_for_network_layer:
      d = *((datagram*) (event.msg));              //retrieve our datagram
      switch(d.type) {
        case DATAGRAM:
          o = malloc(sizeof(TL_OfferElement));     //assign little o as TL_OfferElement
          o->otherHostAddress = d.src;             //assign address
          o->seg = d.payload;                      //assign payload
      }
    case NL_SendingQueueOffer:
      while(EmptyFQ(offer->queue) == 0) {         //while not empty
        e = DequeueFQ(offer->queue);              //pop an element
        o = ((TL_OfferElement*) ValueOfFQE(e));   //assign it to TL_OfferElement

        d.src = thisNetworkAddress;               //resolve source address
        d.dest = o->otherHostAddress;             //resolve destination address

        d.payload = o->seg;  //get the segment from TL_OfferElement as our payload.

        O = malloc(sizeof(NL_OfferElement));      //assign big O as NL_OfferElement
        O->otherHostNeighbourid = NL_TableLookup(d.dest); //get address from lookup table.
        O->dat = d;                        //store out datagram in NL_OfferElement
        EnqueueFQ(NewFQE((void *)O), sendingBuffQueue); //send NL_OfferElement
      }
  }
\end{lstlisting}

The following is the implementation of the signal handling in \texttt{networkLayerRouter()} with unimportant parts ommited.
\begin{lstlisting}
void networkLayerRouter() {
  while(true) {
    Wait(&event, events_we_handle); //block for a new event
    switch(event.type) { //handle events
      case network_layer_allowed_to_send:
        allowedToSendToLL = true;
        break;
      case data_for_network_layer: //received from LL
        d = *((datagram*) (event.msg));
        if(d.dest != thisNetworkAddress) { //forward if not addressed to this router
          O = malloc(sizeof(NL_OfferElement));
          O->otherHostNeighbourid = NL_TableLookup(d.dest); //Lookup the destination
          O->dat = d;
          EnqueueFQ(NewFQE((void *)O), receivedQueue);
          break; //done case.
        }
        break; //break while loop
    }
  }
}
\end{lstlisting}



\subsection{Transport Layer}

\texttt{TL\_OfferElement} is used by the TL to figure out where to process the segment
\begin{lstlisting}
typedef struct {
  networkAddress receiver;  //This contains ``fake'' ip address of the reciever
  payload seg;              //our payload is the msg
} TL_OfferElement;
\end{lstlisting}
\break

Our implementation of the TL with only the most important parts is as follows.
this implementation focuses on the handling of signals, under the hood we use a fifo-queue and the $TL_OfferElement$
\begin{lstlisting}
void fake_transportLayer() {
  while(true) {
    Wait(&event, events_we_handle); //block until a valid signal is recieved.
    switch(event.type) {
      case TL_ReceivingQueueOffer: //Received the TL_ReceivingQueueOffer signal.
        offer = (ConcurrentFifoQueue*) event.msg;
        while(EmptyFQ(offer->queue) == 0) { //Handle incoming messages
          e = DequeueFQ(offer->queue);
          o = *((TL_OfferElement*) (e->val));
        }
        numReceivedPackets++;
        break;
    }
    ... //more signals to be implemented.
  }
}
\end{lstlisting}

\subsection{Part 3}

In part 3 we've extended $events.h$ and included events from $transportLayer.h$ and $appplicationLayer.h$
below is a list of the newly added events.
\begin{itemize}
\item \texttt{TL\_SocketRequest}: Used to assign a new unique socket, TL handles this.

\item \texttt{AL\_Connect}: Called from AL to TL.
We check if the connections array isn't fully occupied and try to open a connection and assign a connection id.

\item \texttt{AL\_Disconnect}: Called from AL to TL. First we make sure that the connection is active, then we de-register the connection in our $connections$ array.
a signal is sent to the recieving host that the connection has been terminated.

\item \texttt{AL\_Send}: Called from AL to TL.

\item \texttt{AL\_Receive}: Called from AL to TL.

\end{itemize}

\subsubsection{TL\_SocketRequest}
\begin{lstlisting}
\end{lstlisting}

\subsubsection{AL\_Connect}
The following is the implementation of $connect$ in AL.
As function arguments, we require a socket and reciever address along with port.
We create the $ALConnReq$ header and send it to the TL for processing,
this also means that the TL is responsible for error handling for this function.
We use $AL\_Connect$ for sending the signal and an implementation of the signal processing is given down below.
\begin{lstlisting}
int connect(TLSocket *socket, networkAddress addr, transPORT port) {
  if(!socket->valid) { return -1; }
  ALConnReq* connReq = malloc(sizeof(ALConnReq));
  connReq->port = port;
  connReq->sock = socket;
  connReq->netAddress = addr;
  connReq->connectionid = CONNECTION_PENDING;
  Signal(AL_Connect, connReq);
  //wait for the connection to be established.
  for(int i = 0; i < 3 && (connReq->connectionid == CONNECTION_PENDING || socket->connections[connReq->connectionid].pending) == 1; i++) { sleep(1); }
\end{lstlisting}

The following is the connection request header.
\begin{lstlisting}
typedef struct {
  transPORT port; //Port that is receiving.
  networkAddress netAddress; //the address to the reciever.
  TLSocket *sock; //The pointer to assign to the address of the returned socket.
  unsigned int connectionid;
} ALConnReq;
\end{lstlisting}

Below is the implementation of the signal $AL\_Connect$.
First we retrieve the msg, cast to our ALConnReq, afterwards, we check and make sure that
there is an available free slot for our connection.
We then construct a $TL\_OfferElement$ with the required information and send it to the NL.
\begin{lstlisting}
case AL_Connect:
  ALConnReq *connReq = (ALConnReq*) event.msg;

  boolean gotAssigned = false;
  for (int i = 0; i < MAX_CONNECTIONS; i++) {
    if (connReq->sock->connections[i].valid == 0) { //check for unused connection.
        connReq->sock->connections[i].valid = 1; //assign that unused connection is used
        connReq->connectionid = i;
        gotAssigned = true;
    }
  }
  if (gotAssigned == false) {
    connReq->connectionid = -1; //update the connReq to return in AL
    break;
  }
  O = (TL_OfferElement*) malloc(sizeof(TL_OfferElement));
  O->otherHostAddress = connReq->netAddress;
  O->segment.is_first = 1;
  O->segment.is_control = 1;
  O->segment.seqMsg = 0;
  O->segment.seqPayload = 0;
  O->segment.senderport = connReq->sock->port;
  O->segment.receiverport = connReq->port;
  O->segment.aux = 0;
  memset(&(O->segment.msg.data), 0, 8);
  EnqueueFQ( NewFQE( (void *) O ), sendingBuffQueue);
break;
\end{lstlisting}



\subsubsection{AL\_Disconnect}

Below is the implementation of the $disconnect$ function, recieved parameters are the socket and the connection id that needs to be terminated.
first we make sure that the connection hasn't already terminated and that the socket is still a valid socket.
We then set the connection to not valid and create a struct $disconnectReq$ which holds our socket and connection id, this is then sent to the TL.
the TL layer then processes the request.
\begin{lstlisting}
//Tries to disconnect the connection specified by the socket and the connection id. returns 0 if sucessfully disconnected otherwise, -1
int disconnect(TLSocket *socket, unsigned int connectionid) {
  ...
 if (socket->connections[connectionid].valid != 1) {
     return -1;
 }
 socket->connections[connectionid].valid = 0;  //register the connection disabled.

 ALDisconnectReq* disconnectReq = malloc(sizeof(ALDisconnectReq));
 disconnectReq->sock = socket;
 disconnectReq->connectionid = connectionid;

 Signal(AL_Disconnect, disconnectReq);
 return 0;
}
\end{lstlisting}


Below is the implementation of the signal $AL\_Disconnect$.
First our ALDisconnectReq is created and derived from the AL via $event.msg$
To disconnect the connection we send to the connected reciever that the connection is no longer valid, to archieve this
we utilize again our $TL\_OfferElement$ to send a control segment to the NL, which passes it on.
\begin{lstlisting}
...
case AL_Disconnect:
  logLine(debug, "AL: Received signal AL_Disconnect\n");

  ALDisconnectReq* disconnectReq = (ALDisconnectReq*) event.msg;

  O = (TL_OfferElement*) malloc(sizeof(TL_OfferElement));
  O->otherHostAddress = disconnectReq->sock->connections[disconnectReq->connectionid].remoteAddress;
  O->segment.is_first = 1;
  O->segment.is_control = 1;
  O->segment.seqMsg = 0;
  O->segment.seqPayload = 0;
  O->segment.senderport = disconnectReq->sock->port;
  O->segment.receiverport = disconnectReq->sock->connections[disconnectReq->connectionid].remotePort; //This one should correspond to the open connection
  O->segment.aux = 0;
  memset(&(O->segment.msg.data), 0, 8);

  EnqueueFQ( NewFQE( (void *) O ), sendingBuffQueue);



  logLine(debug, "AL: DISCONNECTED CONNECTION ID \n");
break;
...
\end{lstlisting}

\subsubsection{AL\_Send}
TODO: implement
\begin{lstlisting}
\end{lstlisting}

\subsubsection{AL\_Receive}
TODO: implement
\begin{lstlisting}
\end{lstlisting}

\subsubsection{AL\_Listen}
Below is the implementation of $listen$ in AL.
Which blocks and listens for incomming packets on the socket.
\begin{lstlisting}
int listen(TLSocket *socket) {
  if (socket->valid == 1) {
      socket->listening |= 1;
      while(socket->listenConnection == CONNECTION_PENDING) {
        sleep(1);
      }
      socket->listening = 0;
      return socket->listenConnection;
  } else {
      return -1;
  }
}
\end{lstlisting}








\hfill \break
