\section{Implementation}

\subsection{Link Layer}
\subsubsection{Neighbours}
The neighbour switching is handled in the selective repeat using a variable called $currentNeighbour$. This variable is set at as early as possible when handling each of the three types of signals handled by selective repeat.\\
The snippets here are trimmed extensively. To get a clear overview, please read the event handling cases in the source file rdt.c. Comments have also been removed.\\
\\~
network\_layer\_ready:\\
The id of the neighbour the packet is intended to be sent to is stored in the msg variable of the event.\\
The fake network layer written for testing gives this value when signalling:
\begin{lstlisting}
Signal(network\_layer\_ready, i);
\end{lstlisting}
And the selective repeat function reads this value when handling the event:
\begin{lstlisting}
currentNeighbour = event.msg;
\end{lstlisting}

frame\_arrival:\\
The id of the neighbour that sent the packet is returned by the function $from\_physical\_layer$.
\begin{lstlisting}
currentNeighbour = from\_physical\_layer(&r);
\end{lstlisting}

timeout:\\
The timer message contains the id of the neighbour it enables.
\begin{lstlisting}
currentNeighbour = ((packetTimerMessage*)event.msg)->neighbour;
\end{lstlisting}

\subsection{Network Layer}
\subsubsection{Datagram}
Datagrams is used for handling messages from/to the transport layer.
The implementation of the datagram and payload is as follows.
\begin{lstlisting}
typedef struct  {
  //First 32bit word
  unsigned char type;
  unsigned short payloadsize;

  //Two next 32-bit words
  networkAddress src;
  networkAddress dest;

  TL_Segment segment;
} datagram;
\end{lstlisting}
The networkAddress is defined as a 32-bit address similarly to IPv4.

\subsubsection{Network Layer Implementation}

The following is the implementation of \texttt{NL\_RoutingTable}, the routing table is used to look up hosts and routers in the network.
Here we've defined a static variable, in the real world the \texttt{NL\_ROUTING\_TABLE\_SIZE} should be incremented and decremented dynamically.
There can be more elements in Routing table than there are neighbours and multiple entries can point to the same neighbour.
\begin{lstlisting}
#define NL_ROUTING_TABLE_SIZE 4
typedef struct {
  networkAddress addresses[NL_ROUTING_TABLE_SIZE];
  neighbourid    neighbourids[NL_ROUTING_TABLE_SIZE];
} NL_RoutingTable;
\end{lstlisting}

The following is the implementation of the signal handling in \texttt{networkLayerRouter()} with unessential parts omitted. The implementation showcases the essential components of receiving from LL and the packet forwarding.
\begin{lstlisting}
void networkLayerRouter() {
  while(true) {
    Wait(&event, events_we_handle); //block for a new event
    switch(event.type) { //handle events
      case network_layer_allowed_to_send:
        allowedToSendToLL = true;
        break;
      case data_for_network_layer: //received from LL
        d = *((datagram*) (event.msg));
        if(d.dest != thisNetworkAddress) { //forward if not addressed to this router
          O = malloc(sizeof(NL_OfferElement));
          O->otherHostNeighbourid = NL_TableLookup(d.dest); //Lookup the destination
          O->dat = d;
          EnqueueFQ(NewFQE((void *)O), receivedQueue);
          break;
        }
        break;
    }
  }
}
\end{lstlisting}

\subsection{Transport Layer}
\subsubsection{types}
The central types of the transport layer are the segments, sockets and connections.\\
\\~
There are two types of segments, message segments and control segments, differentiated by the is\_control flag. The control segments are used to manage establishment and disconnection of connections. The aux field is used to determine the control segment type.\\
The message segments contain message fragments. The first message fragment contains the number of additional fragments in the seqPayload field and the total message byte size in the aux field. The final fragment contains the number of bytes it carries in the aux field.

\begin{lstlisting}
typedef struct {
  unsigned int is_first:1; //True iff the segment is the first of the message.
  unsigned int is_control:1; //true if the segment is a control segment.
  unsigned int seqMsg; //Sequencing number for the messages. No assumptions are made regarding the lower layers and would thereby support random segment arrival order together with seqPayload. (Superficially tested in a closed environment)
  unsigned int seqPayload; //Special case: if is_first is 1, then this value indicates the number of segments the messages are split into, excluding this one.
  unsigned int aux; //is_first: number of bytes of the total message. Used to allocate buffer size. Security? Nope.  For the last message: this is the number of bytes it carries with it.  Otherwise: reserved.
  transPORT senderport; //Port of the application sending the segment.
  transPORT receiverport; //Port of the application receiving the segment.

  payload msg; //Actual bytes carried by the segment.
} TL_Segment;
\end{lstlisting}

Applications in AL obtain sockets from TL. These sockets have a validity flag the applications can check to determine the validity of the requested socket, being 0 if the socket is invalid and cannot be used. The listenConnection value is used when waiting for another host to establish a connection. The port variable is a unique identifier that the client can read if it requested any socket to determine its own port. In addition to these control values, a fixed number of connections is available in each socket.\\
A deprecated value has been omitted. It is still accessed in the code, but not with semantic significance.
\begin{lstlisting}
typedef struct {
  unsigned int valid:1; //Checked upon init. The TL sets this as an indicator for the AL.
  unsigned int listening:2; //This is used to keep track of listening status and when to break out of waiting in the listening function.
  unsigned int listenConnection; //When listening, this value will be set to the id of the connection listened to.
  transPORT port; //The port of the socket. The application can check this value to see which port is has been given if it chose any port.
  TLConnection connections[MAX_CONNECTIONS]; //Connections
} TLSocket;
\end{lstlisting}

The connections have three flags. The valid flag for the validity of the connection and the pending and disconnected flags for connection state. Furthermore, it has address information for the remote host and sequence numbers, one for incrementally enumerating outbound messages and one to determine the sequence number of the next message the application can receive.

\begin{lstlisting}
typedef struct {
  unsigned int valid:1; //Checked on init.
  unsigned int pending:1;
  unsigned int disconnected:1; //This is because you might still want to read messages from the connection after it was disconnected. Also allows for the application to look directly at this.
  networkAddress remoteAddress; //Address of the other machine
  transPORT remotePort; //Port of the other application
  unsigned int outboundSeqMsg; //seqMsg of next message to send.
  unsigned int inboundSeqMsg;  //seqMsg of next message to receive.
  TLMessageBufferLL *msgListHead; //Head of the linked list of messages. If this one has fragmentsRemaining == 0, then its msg can be received by the application.
  TLMessageBufferLL *msgListTail; //Tail of the linked list of messages
} TLConnection;
\end{lstlisting}

\subsubsection{Message Splitting and Merging}
The message splitting and merging are relatively simple. It consists of linear searching through the connection list to find the destination connection and memory copies such as in the following snippet from the splitter, where a part of the message is being copied into the payload part of each segment in a series of segments.
\begin{lstlisting}
memcpy(&(O->segment.msg.data), msgToSplit+(i*MAX_PAYLOAD), MAX_PAYLOAD);
\end{lstlisting}

On the receiving end, this behaviour is mirrored. The copying amount is initially assumed to be the maximum size of a payload, but if it turns out that the fragment is the last one, then the aux field of the segment is used as the copying amount, after which the payload is copied into the message. The sequence number is used to offset into the message in terms of blocks of the maximum payload size.
\begin{lstlisting}
if((o.segment.seqPayload+1)*MAX_PAYLOAD >= i->msgLen) {
  copyamount = o.segment.aux;
}

memcpy((i->msg)+(o.segment.seqPayload*MAX_PAYLOAD), &(o.segment.msg), copyamount);
\end{lstlisting}

\subsubsection{Connections}
When sending the control segments, certain fields are set. This snippet is of the variable assignment when attempting to establish a connection to an application on a remote host.
\begin{lstlisting}
O->otherHostAddress = connReq->netAddress;
O->segment.is_first = 1;
O->segment.is_control = 1;
O->segment.seqMsg = 0;
O->segment.seqPayload = 0;
O->segment.senderport = connReq->sock->port;
O->segment.receiverport = connReq->port; //This one should correspond to the open connection
O->segment.aux = 0;
memset(&(O->segment.msg.data), 0, 8);
\end{lstlisting}

The receiving end then reads the aux field to determine the control segment type, does sanity checks and responds to the control message if necessary.





\hfill \break
